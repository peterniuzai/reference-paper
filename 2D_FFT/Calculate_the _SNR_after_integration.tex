\documentclass{article}
\title{Calculate the SNR after intergration} 
\begin{document}
\author{peter}

\maketitle
\section{Sigma changing}
Assuming we have 2-D Gaussian Noise array, which we can create with Numpy:\\
\\import numpy as np\\
a = np.random.normal(loc = 0 , scale = 1 , size = (1024,1024))\\
\\In array a, the standard deviation is 1, mean is zero. That means in each part of a , the mean and standard deviation should be same or close to it. like we have array b  = a[1:1000,1] , the $\sigma\approx$1,  mean $\approx$0.\\
If we integration an array with shape (N,N) along one axis, the sigma will become:
\begin{equation}
\sigma_i= \sigma_2 \cdot \sqrt{N}
\end{equation}

like c = a.sum(axis = 0). Array c has shape of (1024,) Then the $\sigma$ of c is not 1 anymore, it should be 32.\\

\section{Deduction}
Definition of Standard Deviation:
\begin{equation}
\sigma = \sqrt{\frac{\Sigma_{i=1}^N(x_i-\bar{x})^2}{N-1}}
\end{equation}
For a 2-D array, 
\begin{equation}
\sigma_2= \sqrt{\frac{\Sigma_{i=1}^N\Sigma_{j=1}^N(x_{ij}-\bar{x})^2}{(N-1)\cdot(N-1)}}
\end{equation}
If we integration along one axis,$x_{ij}=\Sigma_{i=1}^N$,then the equation above will becomes:
\begin{equation}
\sigma_i= \sqrt{\frac{\Sigma_{j=1}^N(\Sigma_{i=1}^Nx_{j}-\bar{x})^2}{(N-1)}}
\end{equation}
As $\bar{x}$ should keep zero, the uper iterms under square root should keep the same , denominator under square root is the only change.
Then we get equation (1).

\end{document}
