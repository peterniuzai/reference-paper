\documentclass{article}
\usepackage{enumerate}
\usepackage{url}
\usepackage[colorlinks,linkcolor=blue]{hyperref}
\title{Crab project}
\begin{document}

\maketitle
\section{Data acquistion}
\section{Antenna control}
\subsection{Find the pulsar position}

\begin{enumerate}[step i]
\item Find a pulsar you like from pulsar catalog:\\
\url{http://www.atnf.csiro.au/people/pulsar/psrcat/}\\
Some parameter to choose:
JName: Pulsar has 2 kinds name: B and J
p0 : period \\
p1: derivation of period. As the period is chaning slightly.\\
S400, S1400, S2000 : As a wide band signal, pulsar’s flux is different at different frequency bands. \\
Name, DM , Ra, Dec.

\item Find out the observer’s latitude and longitude from :
http://www.geoastro.de/welcomeEnglish.htm
As leuschner’s latitude is around 38 
\item Then using following website to calculate the object’s altitude and azimuth for the place you stay.We can also calculate this from astropy.  \\
\url{http://www.convertalot.com/celestial_horizon_co-ordinates_calculator.html}
\end{enumerate}

\subsection{Ra ,dec inputs}
\subsection{Antenna response}
\subsection{Real time Camera of Antenna}
We can view the status of Antenna from browser :\\
\url{http://leuschner.berkeley.edu:8080}
\section{Data View and process by useful tools}
Presto and sigproc(sigpyproc) are useful tools to view band pass data or de-dispersion.They are already installed on Crab server.
\subsection{Band pass}

\subsection{ADC data view}
\subsection{De-disperse by sigproc}
\subsection{De-disperse plot}
\section{Analysis}
\subsection{Detection ability of this system}
Our Crab system has a temperature 150K.
\subsection{DM smearing}
\end{document}
